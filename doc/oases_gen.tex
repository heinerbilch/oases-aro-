\section{OASES - General Features}

\subsection{Environmental Models}
\label{oas_env}
    OASES supports all environmental models available in  SAFARI,
i.e.  any  number and combination of isovelocity  fluids,  fluids
with  sound  speed  gradients and  isotropic  elastic  media.  In
addition,  as a new feature any number of transversely  isotropic
layers may be specified  (all with vertical symmetry axis). Further,
media with general dispersion characteristics can be included.

Version 2.0 of OASES in addition allows for stratifications including
an arbitrary number of poro-elastic layers, with the propagation
described by Biot's theory. This modification has been performed by
Morrie Stern \cite{stern} at the University of Texas at Austin,
in collaboration
with Nick Chotiros and Jim tencate at ARL/UT. Nick and Morrie
suggested I include their modifications in the general OASES export
package, for the use and benefit of the general underwater acoustics
community. This is a significant additional capability of OASES, and
the contribution of the Austin group in that regard is highly
appreciated.

\subsubsection{Transversily Isotropic Media}

    A  transversely  isotropic layer is flagged  by  stating  the
usual parameter line for the layer:

\begin{tabular}{cccccccc}
    $D$ &    $c_c$ & $c_s$ & $\alpha_c$ & $\alpha_s$ &  $\rho$ &
$\gamma$ & $ [L] $
\end{tabular}

\noindent with $c_c<0$  as  a  flag.  Here only  the  interface  depth $ D
$ has
significance. The other parameters for the transversely isotropic
layer should then follow in one of two ways, depending on the value
of $c_c$:
\begin{description}
\item[$c_c=-1$:]
          The  medium is specified as a periodic series  of  thin
          layers as per Schoenberg:

	\begin{tabular}{lllllll}
          NC & & & & & &  Number of constituents ($\le 3 $) \\
          $c_c$ & $c_s$ & $\alpha_c$ & $\alpha_s$ &  $\rho$ & $h$ &
                       Speeds, attns., density, fraction \\
          $c_c$ & $c_s$ & $\alpha_c$ & $\alpha_s$ &  $\rho$ & $h$ &
                       Speeds, attns., density, fraction
        \end{tabular}

\item[$c_c = -2$:]
          The 5 complex elastic constants and the density of  the
          transversely  isotropic medium are  specified  directly
          after the flagged layer line:

        \begin{tabular}{ccc}
          Value & Value & Units \\
          Re($C_{11}$) & Im($C_{11}$) &    Pa \\
          Re($C_{13}$) & Im($C_{13}$) &    Pa \\
          Re($C_{33}$) & Im($C_{33}$) &    Pa \\
          Re($C_{44}$) & Im($C_{44}$) &    Pa \\
          Re($C_{66}$) & Im($C_{66}$) &    Pa \\
          $\rho$ & &                       g/cm3
	\end{tabular}
\end{description}
    If option {\bf Z} was specified in the option line, then a slowness
diagram is produced for each transversely isotropic layer.

\subsubsection{Dispersive Media}

For pulse problems where causality is critical, non-zero attenuation
must be accompanied by frequency-dependent wave speeds. Since this is
of main importance for seismic problems with relatively high
attenuation, the specification of dispersive wave speeds has been
limited to elastic media only. A dispersive layer is again flagged by
a negative value of $c_c$:

\begin{description}
\item[$c_c=-3$:] A dispersive layer is specified as followsin the file
{\bf input.dat}:

\begin{tabular}{cccccccc}
    $D$ &    $-3$ & $0$ & $0$ & $0$ &  $\rho$ &
$\gamma$ & $ [L] $ \\	
LTYP & & & & & & &
\end{tabular}

 The only significant parameters for the layer are
the depth $D$, the density $\rho$ and the roughness parameters
$\gamma$ and $L$. The parameter {\bf LTYP} is a type-identifier for
the layer.
The frequency dependence of the
wave speeds and attenuations should be specified in the file {\bf
input.dis}, which may contain several dispersion laws. The file should
contain a block of data in the form of a frequency table, for each value
of {\bf LTYP} specified in {\bf input.dat}, with the first block
corresponding to {\bf LTYP} = 1, the second corresponding to {\bf
LTYP} = 2 etc. The format for each
block is asfollows:

\small
\begin{verbatim}
NF                                         # Number of freqs.
F(1)    CC(1)    CS(1)    AC(1)    AS(1)   # Freq, speeds, atten.
F(2)    CC(2)    CS(2)    AC(2)    AS(2)   # Freq, speeds, atten.
 :        :        :        :        :
F(N)    CC(N)    CS(N)    AC(N)    AS(N)   # Freq, speeds, atten.
\end{verbatim}
\normalsize

The table does not have to be equidistant. OASES will interpolate to
create a table consistent with the frequency sampling specified in
{\bf input.dat}.
\end{description}

\subsubsection{Porous Media}

As an additional feature of the OASES 2.0 environmental model
layers modelled as fluid saturated porous media (Biot model) may be
included with other layer types. The Biot model and its implementation is
described in Appendix~\ref{app_biot}.

A porous sediment layer is flagged by
stating the usual parameter line for the layer in the environmental
data block:

\begin{tabular}{cccccccc}
    $D$ & $c_c$ & $c_s$ & $\alpha_c$ & $\alpha_s$ & $\rho$ &
$\gamma$ & $ [L] $
\end{tabular}

\noindent with negative values for both $c_c$ and  $c_s$.  Only the
interface depth $D$ has significance and must be stated correctly; the
other parameters listed on this data line are dummy.  This line is
immediately followed by a line containing the 13 parameters specifying
the properties of the porous sediment layer in the order:

\begin{tabular}{lllllllllllll}
    $\rho_f$ & $K_f$ & $\eta$ & $\rho_g$ & $K_r$ & $\phi$ & $\kappa$ &
$a$ & $\mu$ & $K$ & $\alpha_s$ & $\alpha_c$ & $c_m$
\end{tabular}

\noindent where

\begin{tabular}{ll}
    $\rho_f$ & is the density of the pore fluid in ${\rm g/cm}^3$ \\
    $K_f$ &	is the bulk modulus of the pore fluid in Pa \\
    $\eta$ &	is the viscosity of the pore fluid in kg/m-s \\
    $\rho_g$ &	is the grain (solid constituent) density in
${\rm g/cm}^3$ \\
    $K_r$ &	is the grain bulk modulus in Pa \\
    $\phi$ &	is the sediment porosity \\
    $\kappa$ &	is the sediment permeability in ${\rm m}^2$ \\
    $a$ &	is the pore size factor in m \\
    $\mu$ &	is the sediment frame shear modulus in Pa \\
    $K$ &	is the sediment frame bulk modulus in Pa \\
    $\alpha_s$	& is the sediment frame shear attenuation \\
    $\alpha_c$	& is the sediment frame bulk attenuation \\
    $c_m$	& is a dimensionless virtual mass parameter
\end{tabular}

   	The sediment frame properties pertain to the drained structure and
are assumed to be dissipative. In particular, for harmonic motion the
frame shear and bulk moduli are taken to be complex in the form
$\tilde{\mu} = \mu (1 + i\mu'),\ \tilde{K} = K(1 + iK')$.
The imaginary parts of the moduli are specified through   $\alpha_s =
20 \pi \mu' \log{e}$  and $\alpha_c = 20 \pi K' \log{e}$  where
$\alpha_s$  corresponds to the attenuation measured in dB/$\Lambda$ of
shear  waves in the sediment frame (as in the data specification for
elastic layers) when the attenuation is low;  $\alpha_c$  is related to
the attenuation of both compression and shear waves.  However, it
should be noted that in contrast to the elastic layer case, the Biot
porous sediment model will yield complex wavespeeds even if the frame
is elastic since dissipation is inherent in the relative motion of the
pore fluid with respect to the frame.

The pore size parameter  $a$  is treated as an empirical constant
which depends on the average grain size and shape; for spherical
grains of diameter  $d$  the value  $a = {\phi d}/{3(1 - \phi)}$ has
been suggested\cite{hovem}.  The virtual mass parameter  $c_m$  (called
the `structure factor' or `tortuosity' by some authors and often
denoted  $\alpha$) is also treated as an empirical constant which
depends on the pore structure of the frame. For moderate frequencies
(long wave length compared to `average pore size') and porosities from
25\% to 50\%, Yavari and Bedford\cite{yavari} have made finite element
calculations which suggest that Berryman's relation  $c_m = 1 +
0.227(1 - \phi)/ \phi$ may be used in the absence of more reliable
data.  More thorough discussions of the material parameters defining
the Biotmodel may be found in other references\cite{biot}.

If the  {\bf K}  option is invoked, then recievers in a porous sediment
layer will output (negative) pore fluid pressure rather than bulk
stress as called for in other fluid or solid layers.  The {\bf Z}
option, which creates a velocity profile plot, shows the zero frequency
limit wavespeeds in porous sediment layers.  Note that at present the
modifications do NOT permit sources in Biot layers.

The following presents a modification of SAFARI-FIP case 3 to
replace the elastic sediment layer by a  poro-elastic layer,

\begin{verbatim}
SAFARI-FIP case 3. Poroelastic.
N C A D J
30 30 1 0
5
  0    0     0     0   0   0   0
  0 1500  -999.999 0   0   1   0 # SVP continuous at z = 30 m
 30 1480 -1490     0   0   1   0
100 -1 -1 0 0 0 0 0              # Cp<0 Cs<0 flag poro-elastic layer
1 2.E9 .001 2.65 9.E9 .4 2.E-9 1.E-5 3.13E8 5.14E9 .8 1.55 1.25
120 1800   600     0.1 0.2 2.0 0

50
0.1 120 41 40
1350 1E8
-1 1 950
0 5 20 1
20 80 12 10
0 120 12 20
40 70 6
\end{verbatim}



\subsubsection{Continuous Sound Speed Profiles}

OASES version 1.7 allows for specifying a continuous sound speed profile
through a flag rather than specifying the negative of the actual value
at the bottom of the layer in the shear speed field as in SAFARI
(OASES still supports this form also). The flag specification is
particularly useful when running through many values of sound speed at
a certain depth, such as for matched field inversion for SVP.

The continuity of the SVP at the {\em bottom of a layer} is activated by the
usual parameter line for the layer:

\begin{tabular}{cccccccc}
    $D$ &    $c_c$ & $c_s$ & $\alpha_c$ & $\alpha_s$ &  $\rho$ &
$\gamma$ & $ [L] $
\end{tabular}

\noindent with $c_s =-999.999 $  as  a  flag, i.e. by setting the shear
speed for the layer to $ -999.999 $. All OASES modules will then set
the sound speed at the bottom of the layer equal to the speed
specified for the top of the next layer below.

As an example, the following is the OAST file \tt saffip3.dat \rm
corresponding to the SAFARI test case 3 with the SVP being continuous
at 30 m depth:

\begin{verbatim}
SAFARI-FIP case 3
N C A D J
30 30 1 0
5
  0    0     0     0   0   0   0
  0 1500  -999.999 0   0   1   0 # SVP continuous at z = 30 m
 30 1480 -1490     0   0   1   0
100 1600   400     0.2 0.5 1.8 0
120 1800   600     0.1 0.2 2.0 0

50
0.1 120 41 40
1350 1E8
-1 1 950
0 5 20 1
20 80 12 10
0 120 12 20
40 70 6
\end{verbatim}

\subsubsection{Stratified Fluid Flow}

OASES version 1.8 allows for computing the field in stratified flow. This
option is only valid in the 2-D versions, handling flow parallel to
the direction of propagation only (downstream or upstream). Flow is
only allowed in {\em isovelocity fluid layers}!

The flow is activated by  specifying  the
usual parameter line for the layer:

\begin{tabular}{cccccccc}
    $D$ &    $c_c$ & $c_s$ & $\alpha_c$ & $\alpha_s$ &  $\rho$ &
$\gamma$ & $ [L] $
\end{tabular}

\noindent with $\alpha_s =-888.888 $  as  a  flag, i.e. by setting the shear
attenuation for the layer to $ -888.888 $. The
flow speed in m/s is in a separate line immidiately following the layer
data line.

The sign convention is {\em positive} for {\em flow from
source to receiver (downstream propagation)} and {\em negative} for
{\em flow from receiver towards source (upstream propagation)}.

As an example, the following is the OAST file \tt saffip1.dat \rm
corresponding to the SAFARI test case 1 with
a flow in the water column of 5 m/s towards the source (upstream propagation):

\begin{verbatim}
SAFARI FIP case 1. Flow -5 m/s.
J N I T
5 5 1 0
4
  0    0    0     0    0     0   0
  0 1500    0     0 -888.888 1.0 0
-5.0                              # flow speed 5 m/s towards source
100 1600  400     0.2  0.5   1.8 0
120 1800  600     0.1  0.2   2.0 0
95
100 100 1 1
200 1E8
-1 1 1000
0 5.0 20 1.0
20 80 12 10
\end{verbatim}



