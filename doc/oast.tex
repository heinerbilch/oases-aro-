\section{    OAST: OASES Transmission Loss Module}

    Except for the specification of frequencies, 
OAST  is downward compatible with SAFARI-FIP  Version 
3.0  and higher, and therefore supports all options and  features 
described in the SAFARI manual. In addition to improved speed and 
stability, OAST offers several new options.

\subsection{Input Files for OAST}

The input files for OAST is structured in 12 blocks, as outlined in
Tables\,\ref{tab:fipI} and \ref{tab:fipII}. In the following we
describe the significance of the various blocks, with particular
emphasis on differences between SAFARI-FIP and OAST. 

\begin{table}
\begin{center}
\small
\begin{tabular}{|l|l|c|c|}
\hline \hline
Input parameter & Description & Units & Limits \\
\hline \hline
\multicolumn{4}{|l|}{\bf BLOCK I: TITLE } \\
\hline
TITLE & Title of run  & - & $\leq$ 80 ch. \\
\hline
\multicolumn{4}{|l|}{\bf BLOCK II: OPTIONS} \\
\hline
A B C $\cdots$ & Output options & - & $\leq$ 40 ch. \\
\hline
\multicolumn{4}{|l|}{\bf BLOCK III: FREQUENCIES} \\
\hline
FR1,FR2,NF,COFF,[V] & FR1: First frequency & Hz & $>0$ \\
	& FR2: Last frequency & Hz & $>0$ \\
	& NF: Number of frequencies & & $>0$ \\
 	& COFF: Integration contour offset & dB/$\Lambda$ & COFF$\geq 0$ \\
	& V: Source/receiver velocity (only for option d) \\
\hline
\multicolumn{4}{|l|}{\bf BLOCK IV: ENVIRONMENT} \\
\hline
NL 	& Number of layers, incl. halfspaces	& - & NL$\geq 2$  \\
D,CC,CS,AC,AS,RO,RG,CL & D: Depth of interface. & m & - \\
.	& CC: Compressional speed & m/s & CC$\geq 0$ \\
.	& CS: Shear speed & m/s & - \\
.	& AC: Compressional attenuation & dB/$\Lambda$ & AC$\geq 0$ \\
.	& AS: Shear attenuation & dB/$\Lambda$ & AS$\geq 0$ \\
	& RO: Density 	& g/cm$^{3}$ & RO$\geq 0$ \\
	& RG: RMS value of interface roughness & m & - \\
	& CL: Correlation length of roughness & m & CL$>0$ \\
	& M:  Spectral exponent &   & > 1.5 \\
\hline
\multicolumn{4}{|l|}{\bf BLOCK V: SOURCES} \\
\hline
SD,NS,DS,AN,IA,FD,DA & SD: Source depth (mean for array) & m & - \\
	& NS: Number of sources in array & - &  NS$>0$ \\
	& DS: Vertical source spacing	& m & DS$>0$ \\
	& AN: Grazing angle of beam & deg & - \\
	& IA: Array type & - & $1 \leq $IA$\leq 5$ \\
	& FD: Focal depth of beam & m & FD$\neq$SD \\
        & DA: Dip angle. (Source type 4). & deg & - \\
\hline
\multicolumn{4}{|l|}{\bf BLOCK VI: RECEIVERS} \\
\hline
RD1,RD2,NR,IR & RD1: Depth of first receiver & m & - \\
	& RD2: Depth of last receiver  & m & RD2$>$RD1 \\
	& NR: Number of receivers & - & NR$>0$ \\
	& IR: Plot output increment & - & IR$\geq0$ \\
\hline
\multicolumn{4}{|l|}{\bf BLOCK VII: WAVENUMBER SAMPLING} \\
\hline
CMIN,CMAX & CMIN: Minimum phase velocity & m/s & CMIN$>0$ \\
	& CMAX: Maximum phase velocity & m/s & - \\
NW,IC1,IC2 & NW: Number of wavenumber samples & - & NW$=2^{M}, -1 (auto)$ \\
	& IC1: First sampling point & - & IC1$\geq 1$ \\
	& IC2: Last sampling point & - & IC2$\leq$NW \\
\hline
\end{tabular}
\end{center}
\caption{Layout of OAST input files: I. Computational parameters.
	\label{tab:fipI} }
\end{table} 

\begin{table}
\begin{center}
\small
\begin{tabular}{|l|l|c|c|}
\hline \hline
Input parameter & Description & Units & Limits \\
\hline \hline
\multicolumn{4}{|l|}{\bf BLOCK VIII: RANGE AXES} \\
\hline
RMIN,RMAX,RLEN,RINC & RMIN: Minimum range on plots & km & - \\
	& RMAX: Maximum range on plots & km & - \\
	& RLEN: Length of x-axis for all plots & cm & RLEN$>0$ \\
	& RINC: Distance between tick marks & km & RINC$>0$ \\
\hline
\multicolumn{4}{|l|}{\bf BLOCK IX: TRANSMISSION LOSS AXES (Only for
Options A,D,T) } \\
\hline
TMIN,TMAX,TLEN,TINC & TMIN: Minimum transmission loss & dB & - \\
 & TMAX: Maximum transmission loss & dB & - \\
	& TLEN: Length of vertical TL axes & cm & TLEN$>0$ \\
	& TINC: Distance between tick marks & dB & TINC$>0$ \\
\hline
\multicolumn{4}{|l|}{\bf BLOCK X: DEPTH AXES (Only for Options C,D)} \\
\hline
DUP,DLO,DLN,DIN & DUP: Min. depth for plots & m & - \\
         & DLO: Max. depth for plots & m & - \\
	 & DLN: Length of depth axes & cm & DCLN$>0$ \\
	& DIN: Distance between tick marks & m & DCIN$>0$ \\
\hline
\multicolumn{4}{|l|}{\bf BLOCK XI: CONTOUR LEVELS (Only for Option C,f)} \\
\hline
ZMIN,ZMAX,ZINC & ZMIN: Minimum contour level & dB & - \\
(	& ZMAX: Maximum contour level & dB & - \\
	& ZINC: Contour level increment & dB & ZINC$>0$ \\
\hline
\multicolumn{4}{|l|}{\bf BLOCK XII: SVP AXES (Only for Option Z)} \\
\hline
VLEF,VRIG,VLEN,VINC & VLEF: Wave speed at left border & m/s & - \\
DVUP,DVLO,DVLN,DVIN & VRIG: Wave speed at right border & m/s & - \\
                & VLEN: Length of wave speed axis & cm & VLEN$>0$ \\
		& VINC: Wave speed tick mark distance & m/s & VINC$>0$ \\
		& DVUP: Depth at upper border  & m & - \\
		& DVLO: Depth at lower border  & m & - \\
		& DVLN: Length of depth axis   & cm & DVLN$>0$ \\
		& DVIN: Depth axis tick mark distance & m & DVIN$>0$ \\
\hline
\end{tabular}
\end{center}
\caption{Layout of OAST input files: II. Plot parameters.
	\label{tab:fipII} }
\end{table} 

\subsubsection{Block I: Title}

The title printed on all graphic output generated by OAST.

\subsubsection{Block II: OAST options}

In addition to supporting the SAFARI options described in \cite{hs:saf}, 
OAST supports a wide suite of new options.
\begin{itemize}
	     \item[{\bf A}] Depth-averaged transmission loss plotted
		for each of the selected field parameters. 
		The averaging is performed over
		the specified number of receivers (block VI).
	     \item[{\bf C}] Range-depth contour plot for transmission
                loss. Only allowed for one field parameter
		at a time.
    \item[{\bf F}] In versions earlier than 2.3a  Filon-FFT is applied to evaluate the wavenumber
integrals instead of the default FFP. However, this overestimates the transmission loss and provides no benefit for transmission loss calculations if the sampling criteria are satisfied. In fact it yields up to 3 dB error for automatic sampling and has therefore been disabled in the newer versions of OAST.
    \item[{\bf G}] Rough interfaces are assumed to be characterized by
a Goff-Jordan power spectrum rather than the default Gaussian. 
	     \item[{\bf H}] Horizontal velocity calculated.
	     \item[{\bf I}] Hankel transform integrands are plotted
		for each of the selected field parameters.
	     \item[{\bf J}]	Complex integration contour. The contour is shifted
		into the upper halfpane by an offset controlled by the 
		input parameter COFF (Block III).
\item[{\bf K}]
      Computes the bulk pressure. In elastic media the bulk pressure only
      has contributions from the compressional potential. In fluid
      media the bulk pressure is equal to the acoustic pressure.
      Therefore for fluids this option yields the negative of the result
      produced by option N or R.  

	     \item[{\bf L}] Linear vertical source array.
	     \item[{\bf N}] Normal stress $\sigma_{zz}$ ($=-p$ in fluids) calculated.
	     \item[{\bf P}] Plane geometry. The sources will be line-sources
		instead of point-sources as used in the default
		cylindrical geometry.
    \item[{\bf R}] Computes the radial normal stress $\sigma_{rr}$ (or
		$\sigma_{xx}$ 	for plane geometry). 
    \item[{\bf S}] Computes the stress equivalent of the shear potential in
         elastic media. This is an angle-independent measure, proportional to 
         the shear potential, with no contribution from the compressional 
         potential (in contrast to shear stress on a particular plane). 
        	For fluids this option yields zero.
	     \item[{\bf T}] Transmission loss plotted as function of
		range for each of the selected field
		parameters.
	     \item[{\bf V}] Vertical velocity calculated.
	     \item[{\bf Z}] Plot of velocity profile.
    \item[{\bf a}]     Angular spectra of the integration kernels are plotted. 
          A  $0 - 90^{\circ}$ axis is automatically selected  representing 
          the  grazing  angle  (  $0^{\circ}$  corresponds  to  horizontal 
          propagation ). NOTE: The same wavenumber corresponds to 
          different  grazing  angles  in  different  media!.  The 
          vertical  axis is selected automatically,  representing 
          the  angular  density  (as opposed  to  the  wavenumber 
          density for integrand plots ( option I ).
    \item[{\bf b}] Solves the depth-separated wave equation with the
lowermost  interface condition expressed in terms of a complex
reflection coefficient. The reflection coefficient must be tabulated in a input file \tt input.trc \rm
which may either be produced from experimental data or by the
reflection coefficient module OASR as described on
Page\,\pageref{trc-form}. See also there for the file format.
The lower halfspace must be specified as vacuum and the last layer as
an isovelocity fluid without sources for this option. Add dummy layer
if necessary. Further, the
frequency sampling must be consistent. Therefore, if this option is
combined with option \tt f \rm, the input file must have cosistent logarithmic
sampling. Using \tt OASR \rm this is optained by using option \tt C
\rm with the same minimum and maximum frequencies, and number of frequencies.
Note: Care should be taken using this option with a complex
integration contour, option \tt J \rm. The tabulated reflection
coefficient must clearly correspond to the same imaginary wavenumber
components for \tt OAST \rm to yield proper results. \tt OASR \rm calculates
the reflection coefficient for real horizontal wavenumbers.
    \item[{\bf c}]     Contours   of  integration  kernels  as   function   of 
          horizontal  wavenumber  (abcissa)  and  receiver  depth           
          (ordinate). The horizontal wavenumber axis is  selected           
          automatically,  whereas  the  depth  axis  is   plotted           
          according  to  the parameters given for option  C.  The           
          contour levels are determined automatically.
\item[{\bf d}] Source/receiver dynamics. OAST v 1.7 handles the problem of
source and receiver moving through the waveguide at the same speed and
direction. The velocity projection V onto the line connecting source
and  receiver must be specified in Block III, as shown in
table~\ref{tab:fipI}. Since source and receiver are moving at identical
speeds there is no Doppler shift, but the Green's function is
different from the static one, as described by Schmidt and
Kuperman\cite{sk:jasa94}.
    \item[{\bf f}] A full Hankel transform integration scheme is used
for low values of $ kr $ and tapered into the FFP integration used for
large $ kr $. The compensation is achieved at very low computational
cost and is recommended highly for cases where the near field is
needed.   
    \item[{\bf o}] Contours of transmission loss plotted vs frequency
and range, i.e. the traditional 'optimum frequency' contour plots. 
Requires NFREQ $> 1$ (see below). A logarithmic frequency
axis is assumed for this option. Requires ZMIN, ZMAX and ZINC to be
specified in Block XII (same contour levels as for option {\bf C}
which may be specified simultaneously). 
    \item[{\bf g}] Rough interfaces are assumed to be characterized by
a Goff-Jordan power spectrum rather than the default Gaussian (Same as G). 
    \item[{\bf l}] User defined source array. This new option is
similar to option {\bf L} in the sense that that it introduces a
vertical source array of time delayed sources of identical type. However,
this option allows the depth, amplitude and  delay time to be be
specified individually for each source in the array. The source data
should be provided in a separate file, {\bf input.src}, in the format
described below in Section~\ref{oastsou}.
    \item[{\bf s}]     Outputs  the  mean  field  discontinuity  at  a   rough 
          interface  to  the file 'input'.rhs for  input  to  the 
          reverberation model OASS.
    \item[{\bf t}] Solves the depth-separated wave equation with the
top  interface condition expressed in terms of a complex
reflection coefficient. The reflection coefficient must be tabulated in a input file \tt input.trc \rm
which may either be produced from experimental data or by the
reflection coefficient module \tt OASR \rm as described on
Page\,\pageref{trc-form}. See also there for the file format.
The upper halfspace must be specified as vacuum and the first layer as
an isovelocity fluid without sources for this option. Add dummy layer
if necessary. Further, the
frequency sampling must be consistent. Therefore, if this option is
combined with option \tt f \rm, the input file must have cosistent logarithmic
sampling. Using \tt OASR \rm this is optained by using option \tt C
\rm with the same minimum and maximum frequencies, and number of frequencies.
Note: Care should be taken using this option with a complex
integration contour, option \tt J \rm. The tabulated reflection
coefficient must clearly correspond to the same imaginary wavenumber
components for \tt OAST \rm to yield proper results. \tt OASR \rm calculates
the reflection coefficient for real horizontal wavenumbers.
    \item[{\bf \#}] Number $(1-4)$ specifying the source type
       (explosive, forces, moment) as described in Section~\ref{oastsout}  
\end{itemize}

\subsubsection{Block III: Frequencies}

A frequency loop has been incorporated in OAST to allow for computation
of transmission loss over a wide frequency band in one run. The frequency
specification (Block III in SAFARI Manual) has therefore been changed to:

\begin{verbatim}
FR1 FR2 NF COFF [V]
\end{verbatim}

where FR1 and FR2 are the minimum and maximum frequencies, respectively.
NF is the number of frequencies, spaced equidistantly between FR1
and FR2, except if option {\bf f} was specified; then the frequencies
will be spaced logarithmically. COFF is the complex wavenumber
integration 
contour offset. To be specified in
		$dB/\lambda$, where $\lambda$ is the wavelength at the source 
		depth SD. As only the horizontal part of
		the integration contour is considered, this parameter
		should not be chosen so large, that the amplitudes
		at the ends of the integration interval become 
		significant. In lossless cases too small values
		will give sampling problems at the normal modes
		and other singularities. For intermediate values,
		the result is independent of the choice of COFF,
		but a good value to choose is one that gives 60 dB
		at the longest range considered in the FFT, i.e.
                \begin{displaymath}
		\mbox{COFF} = \frac{60 \ast \mbox{CC(SD)}}{( \mbox{FREQ} \ast \mbox{R}_{max} )}
		\end{displaymath} 
       		where the maximum FFT range is
       		\begin{displaymath}
		\mbox{R}_{max} = \frac{\mbox{NP}}{\mbox{FREQ}\ast(1/\mbox{CMIN} - 1/\mbox{CMAX})}
		\end{displaymath}
		This value is the default which is applied if COFF
		is specified to 0.0.

The optional parameter V is the identical speed of source and receiver
relative to the medium, 
projected onto the radial vector connecting them.
This parameter is only used for option {\bf d}.

\subsubsection{Block IV: Environmental Model}

OAST supports all the environmental models allowed for SAFARI as well as the
ones described above in Section~\ref{oas_env}.  The significance of
the standard environmental parameters is as follows
\begin{itemize}	
		\item[NL:]	Number of layers, including the upper and lower
		half-spaces. These should Always be included,
		even in cases where they are vacuum.

		\item[D:]	Depth in $m$ of upper boundary of layer or
		halfspace. The reference depth can be choosen
		arbitrarily, and D() is allowed to be negative.
		For layer no. 1, i.e. the upper half-space, this
     		parameter is dummy.

		\item[CC:]	Velocity of compressional waves in $m/s$.
        	If specified to 0.0, the layer or half-space is
		vacuum.

		\item[CS:]	Velocity of shear waves in m/s.
 		If specified to 0.0, the layer or half-space is fluid.
                If CS()$< 0$, it is the compressional velocity at bottom of
		layer, which is treated as fluid with $1/c(z)^{2}$ linear.

		\item[AC:]   Attennuation of compressional waves in 
		$dB/\lambda$. If the layer is fluid, and AC() is specified to
		0.0, then an imperical water attenuation is
		used (Skretting \& Leroy).

		\item[AS:]   Attenuation of shear waves in $dB/\lambda$

		\item[RO:]   Density in $g/cm^{3}$.

		\item[RG:] RMS roughness of interface in $m$. RG(1) is
		dummy. If RG$<0$ it represents the negative of the RMS
		roughness, and the associated correlation length CL
		and spectral exponment M should follow. If RG$>0$ the 
                correlation length is
		assumed to be infinite.  
                \item[CL:] Roughness
		correlation length in m.  

                \item[M:] Spectral exponent of the power spectrum as
                defined by Turgut \cite{Turgut_97}, with $1.5 <
                \mbox{M} \le 2.5 $ for realistic surfaces, with
                $\mbox{M} =1.5 $ corresponding to the highest
                roughness, and $\mbox{M}=2.5$ being a very smooth
                variation. For 2-D Goff-Jordan surfaces, the fractal
                dimension is $ \mbox{D} = 4.5 - \mbox{M} $
                Insignificant
                for Gaussian spectrum (option {\bf g} not specified)
                but a  value must
                be given.
                \end{itemize}


\subsubsection{Block V: Sources}

    OAST supports the same sources as  SAFARI-FIP,  i.e 
explosive sources in fluids or solids or vertical point forces in 
solids  (option  X).  Multible sources in a  vertical  array  are 
supported. If sources with horizontal directionality are desired, 
the 3-dimensional version OASP3D must be used. The significance of the
source parameters are as follows
\begin{itemize}
		\item[SD:] Source depth in $m$. If option 'L' has been 
		specified, then SD defines the mid-point of the vertical 
		source array.
		\item[NS:]	Number of sources in the array.
		\item[DS:] Source spacing in $m$.
		\item[AN:]  Specifies the nominal grazing ANG of the 
		generated beam in degrees. $ANG>0$ corresponds to
		downward propagation.
		\item[IA:]	Array type
			\begin{enumerate}
	        	\item Rectangular weighted array
		 	\item Hanning weighted array
		 	\item Hanning weighted focusing array
			\item Gaussian weighted array
			\item Gaussian weighted focusing array
			\end{enumerate}
		\item[FD:] Focal depth in $m$ for an array of type 3 and 5.
       		\item[DA:] Dip angle in degrees for dip-slip sources (type 4).
		\end{itemize}


\noindent {\bf Source Types}
\label{oastsout}  

As in SAFARI the default source type in OAST 
is an explosive type compressional source. In addition to the optional
vertical point force, and axisymmetric seismic moment source has been
added to OAST. The source type is specified by a number $(1-3)$ in the
option field (line 2). The translation is as follows:
\begin{enumerate}
\item Explosive source (default) normalized to unit pressure at 1 m distance.
\item Vertical point force with amplitude 1 N.
\item Horizontal (in-plane) point force with amplitude 1 N.
\item Dip-slip source with seismic moment 1 Nm. Dip angle specified in 
      degrees in block V, following the other parameters.
\item Omnidirectional seismic moment source representing explosive source. Same as type 1, but all three force dipoles have seismic moment 1 Nm.
\end{enumerate}

\noindent {\bf Source Normalization}

    In  SAFARI,  the source strength was normalized to yield unit
pressure (in Pa) 
at a distance of 1 m from the source (for 
solids the negative of the normal stress 1 m below the source). 

    In  OASES, the same source normalization has been  maintained 
for point sources (explosive sources) in fluid media.
    For solid media, however, the sources are normalized to  unit 
volume (1 m$^3$) injection for explosive sources and unit force 
1 N for point sources or 1 N/m for line sources.

\noindent {\bf User defined Source Arrays}
\label{oastsou}

Version 1.6 of OAST has been upgraded to allow a user-defined source
array through option {\bf l}. 

Option {\bf l} is intended
for general physical arrays with uneven spacing or special shadings,
As for the built-in arrays, such user-defined arrays may be present in
fluid as well as elastic media. The array definition should be given
in the file {\bf input.src} in the following format

\small
\begin{verbatim}
LS
SDC(1)  SDELAY(1)  SSTREN(1)    # Depth (m), Delay (s), Amplitude
SDC(2)  SDELAY(2)  SSTREN(2) 
SDC(3)  SDELAY(3)  SSTREN(3) 
  :        :          :
  :        :          :
SDC(LS) SDELAY(LS) SSTREN(LS) 
\end{verbatim}
\normalsize

\subsubsection{Block VI: Receivers}

The default specification of the receiver depths is the same as for
SAFARI, i.e. through the parameters RD1, RD2, NR and IR in Block VI, with
\begin{itemize}
\item[RD1]  Depth of uppermost receiver in m 
\item[RD2]  Depth of lowermost receiver in m 
\item[NR] Number of receiver depths 
\item[IR]   Depth increment for options I and T
\end{itemize}

By default, the NR receivers are placed equidistantly in the vertical.

\noindent {\bf Non-equidistant Receiver Depths}

In OASES the receiver depths can optionally be specified individually.
The parameter NR is used as a flag for this option. Thus, if NR $< 0$
the number of receivers is interpreted as --NR, with the individual
depths following immidiately following Block VI. As an example, SAFARI
FIP case 2 with receivers at 100, 105 and 120 m is run with the
following data file: 

\small
\begin{verbatim}
SAFARI-FIP case 2. 
N I T J Z
30 30 1 0
5
  0    0     0 0   0   0   0
  0 1500 -1480 0   0   1   0
 30 1480 -1490 0   0   1   0
100 1600   400 0.2 0.5 1.8 0
120 1800   600 0.1 0.2 2.0 0
50
100 100 -3 1		       # 3 receiver depths
100.0 105.0 120.0              # Receiver depths in meters
700 1E8  
1024 1 512
 0  5 20  1
20 80 12 10
1450 1550 10 25
   0  100 10 20
\end{verbatim}
\normalsize 


\subsubsection{Block VII: Wavenumber Integration}

This block specifies the wavenumber sampling in the standard SAFARI
format. The critical issues involved in selecting the wavenumber
sampling is described in the SAFARI manual \cite{hs:saf}, but even
more detailed in {\em Computational Ocean Acoustics} \cite{jkps}. 
The structure of this input block is  as follows:
\begin{itemize}
		\item[CMIN:]   Minimum phase velocity in m/s. Determines the
		upper limit of the truncated horizontal wave-
		number space:
			\begin{displaymath}
			k_{max} = \frac{2\pi \ast \mbox{FREQ}}{\mbox{CMIN}}
			\end{displaymath}
		\item[CMAX:]	Maximum phase velocity in m/s. Determines the
		lower limit of the truncated horizontal wave-
		number space:
			\begin{displaymath}
			k_{min} = \frac{2\pi \ast \mbox{FREQ}}{\mbox{CMAX}}
			\end{displaymath}
		In plane geometry ( option P ) CMAX may be specified as 
		negative. In this case, the negative wavenumber spectrum
		will be included with $k_{min}=-k_{max}$, yielding correct 
		solution also at zero range. In contrast to SAFARI,
OAST allows for complex contour integration (option J) in this case.
		\item[NW:]	Number of sampling points in wavenumber space.
		Should be an integer power of 2, i.e. NWN$=2^{m}$.
		The sampling points are placed equidistantly
		in the truncated wavenumber space determined
		by CMIN and CMAX.

		\item[IC1:]  Number of the first sampling point, where the
		calculation is to be performed. If IC1$>1$, 
		then the Hankel transform is zeroed for sampling 
		points 1,2$\ldots$IC1-1, and the discontinuity
		is smoothed.

		\item[IC2:]	Number of the last sampling point where the 
		calculation is to be performed. If IC2$<$NWN,
		then the Hankel transform is zeroed for sampling
		points IC2+1,$\ldots$NW, and the discontinuity
		is smoothed by Hermite polynomial extrapolation.
		\end{itemize}

\noindent {\bf Automatic wavenumber sampling}

    OAST Version  1.5  has been supplied with  an  automatic  sampling 
feature,  making  it possible for inexperienced users  to  obtain 
correct   answers  in  the  first  attempt  without   the   usual 
convergence  testing.  The  automatic sampling  is  activated  by 
specifying the parameter NW = -1 and it automatically  activates 
the  complex wavenumber integration contour even though option  $J$ 
may  not have been specified. The parameters IC1 and IC2 have  no 
effect if the automatic sampling is selected.

As an example, to run the SAFARI-FIP case 2 problem with automatic
sampling, change the data file as  follows:

\small
\begin{verbatim}
SAFARI-FIP case 2. Auto sampling.
N I T J Z
30 30 1 0
5
  0    0     0 0   0   0   0
  0 1500 -1480 0   0   1   0
 30 1480 -1490 0   0   1   0
100 1600   400 0.2 0.5 1.8 0
120 1800   600 0.1 0.2 2.0 0
50
100 100 1 1
1300 1E8                       # CMIN = 1300
-1 0 0                         # NW   =   -1
 0  5 20  1
20 80 12 10
1450 1550 10 25
   0  100 10 20
\end{verbatim}
\normalsize 

\subsection{Execution of OAST}

    As  for  SAFARI,  filenames  are  passed  to  the  code   via 
environmental parameters. In Unix systems a typical command  file 
{\bf oast} (in  \$HOME/oases/bin) is:

\small
\begin{verbatim}
    #                            the pound sign invokes the C-shell 
    setenv FOR001 $1.dat       # input file 
    setenv FOR002 $1.src       # Source array input file
    setenv FOR019 $1.plp       # plot parameter file
    setenv FOR020 $1.plt       # plot data file  
    setenv FOR023 $1.trc       # reflection coefficient table (input)
    setenv FOR028 $1.cdr       # contour plot parameter file 
    setenv FOR029 $1.bdr       # contour plot data file 
    setenv FOR045 $1.rhs       # file for scattering discontinuities  
    oast2                      # executable
\end{verbatim}
\normalsize

    After preparing a data file with the name {\bf input.dat}, OAST  is 
executed by the command:

    $>$ {\bf oast input}

\subsection{Graphics}  

    Command files are provided in a path directory for generating 
the graphics.

\noindent    To generate curve plots, issue the command:

    $>$ {\bf mplot input}

\noindent    To generate contour plots, issue the command:

    $>$ {\bf cplot input}

